%% LyX 2.3.7 created this file.  For more info, see http://www.lyx.org/.
%% Do not edit unless you really know what you are doing.
\documentclass[journal,article,submit,pdftex,moreauthors]{Definitions/mdpi}
\usepackage[utf8]{inputenc}
\usepackage{float}
\usepackage{url}
\usepackage{amsmath}

\makeatletter

%%%%%%%%%%%%%%%%%%%%%%%%%%%%%% LyX specific LaTeX commands.

\Title{Creating features using particle swarm optimization}

\TitleCitation{Creating features using particle swarm optimization}

\Author{Ioannis G. Tsoulos$^{1,*}$, Alexandros Tzallas$^{2}$}

\AuthorNames{Ioannis G. Tsoulos; Alexandros Tzallas}

\AuthorCitation{Tsoulos, I.G.; Tzallas A}


\address{$^{1}$\quad{}Department of Informatics and Telecommunications,
University of Ioannina, Greece;itsoulos@uoi.gr\\
$^{2}\quad$Department of Informatics and Telecommunications, University
of Ioannina, Greece;tzallas@uoi.gr}


\corres{Correspondence: itsoulos@uoi.gr; }


\abstract{The problem of data classification or data fitting is widely applicable
in a multitude of scientific areas, and for this reason a number of
machine learning models have been developed. However, in many cases,
these models present problems of overfitting and cannot generalize
satisfactorily to unknown data. Furthermore, in many cases, many of
the features of the input data do not contribute to learning, or there
may even be hidden correlations between the features of the dataset.
The current article proposes a method based on Particle Swarm Optimization
and Grammatical Evolution, that creates artificial features from the
original ones. In addition, this new technique utilizes penalty factors
to limit the generated features to a range of values to make training
machine learning models more efficient. The method was tested on a
series of well - known datasets from the relevant literature and was
tested against a series of widely used machine learning models.}


\keyword{Particle swarm optimization, Grammatical Evolution, Evolutionary
techniques, Stochastic methods.}

\DeclareTextSymbolDefault{\textquotedbl}{T1}
%% Because html converters don't know tabularnewline
\providecommand{\tabularnewline}{\\}
\floatstyle{ruled}
\newfloat{algorithm}{tbp}{loa}
\providecommand{\algorithmname}{Algorithm}
\floatname{algorithm}{\protect\algorithmname}

%%%%%%%%%%%%%%%%%%%%%%%%%%%%%% Textclass specific LaTeX commands.
\newenvironment{lyxcode}
	{\par\begin{list}{}{
		\setlength{\rightmargin}{\leftmargin}
		\setlength{\listparindent}{0pt}% needed for AMS classes
		\raggedright
		\setlength{\itemsep}{0pt}
		\setlength{\parsep}{0pt}
		\normalfont\ttfamily}%
	 \item[]}
	{\end{list}}

%%%%%%%%%%%%%%%%%%%%%%%%%%%%%% User specified LaTeX commands.
%  LaTeX support: latex@mdpi.com 
%  For support, please attach all files needed for compiling as well as the log file, and specify your operating system, LaTeX version, and LaTeX editor.

%=================================================================


% For posting an early version of this manuscript as a preprint, you may use "preprints" as the journal and change "submit" to "accept". The document class line would be, e.g., \documentclass[preprints,article,accept,moreauthors,pdftex]{mdpi}. This is especially recommended for submission to arXiv, where line numbers should be removed before posting. For preprints.org, the editorial staff will make this change immediately prior to posting.

%--------------------
% Class Options:
%--------------------
%----------
% journal
%----------
% Choose between the following MDPI journals:
% acoustics, actuators, addictions, admsci, adolescents, aerospace, agriculture, agriengineering, agronomy, ai, algorithms, allergies, alloys, analytica, animals, antibiotics, antibodies, antioxidants, applbiosci, appliedchem, appliedmath, applmech, applmicrobiol, applnano, applsci, aquacj, architecture, arts, asc, asi, astronomy, atmosphere, atoms, audiolres, automation, axioms, bacteria, batteries, bdcc, behavsci, beverages, biochem, bioengineering, biologics, biology, biomass, biomechanics, biomed, biomedicines, biomedinformatics, biomimetics, biomolecules, biophysica, biosensors, biotech, birds, bloods, blsf, brainsci, breath, buildings, businesses, cancers, carbon, cardiogenetics, catalysts, cells, ceramics, challenges, chemengineering, chemistry, chemosensors, chemproc, children, chips, cimb, civileng, cleantechnol, climate, clinpract, clockssleep, cmd, coasts, coatings, colloids, colorants, commodities, compounds, computation, computers, condensedmatter, conservation, constrmater, cosmetics, covid, crops, cryptography, crystals, csmf, ctn, curroncol, currophthalmol, cyber, dairy, data, dentistry, dermato, dermatopathology, designs, diabetology, diagnostics, dietetics, digital, disabilities, diseases, diversity, dna, drones, dynamics, earth, ebj, ecologies, econometrics, economies, education, ejihpe, electricity, electrochem, electronicmat, electronics, encyclopedia, endocrines, energies, eng, engproc, ent, entomology, entropy, environments, environsciproc, epidemiologia, epigenomes, est, fermentation, fibers, fintech, fire, fishes, fluids, foods, forecasting, forensicsci, forests, foundations, fractalfract, fuels, futureinternet, futureparasites, futurepharmacol, futurephys, futuretransp, galaxies, games, gases, gastroent, gastrointestdisord, gels, genealogy, genes, geographies, geohazards, geomatics, geosciences, geotechnics, geriatrics, hazardousmatters, healthcare, hearts, hemato, heritage, highthroughput, histories, horticulturae, humanities, humans, hydrobiology, hydrogen, hydrology, hygiene, idr, ijerph, ijfs, ijgi, ijms, ijns, ijtm, ijtpp, immuno, informatics, information, infrastructures, inorganics, insects, instruments, inventions, iot, j, jal, jcdd, jcm, jcp, jcs, jdb, jeta, jfb, jfmk, jimaging, jintelligence, jlpea, jmmp, jmp, jmse, jne, jnt, jof, joitmc, jor, journalmedia, jox, jpm, jrfm, jsan, jtaer, jzbg, kidney, kidneydial, knowledge, land, languages, laws, life, liquids, literature, livers, logics, logistics, lubricants, lymphatics, machines, macromol, magnetism, magnetochemistry, make, marinedrugs, materials, materproc, mathematics, mca, measurements, medicina, medicines, medsci, membranes, merits, metabolites, metals, meteorology, methane, metrology, micro, microarrays, microbiolres, micromachines, microorganisms, microplastics, minerals, mining, modelling, molbank, molecules, mps, msf, mti, muscles, nanoenergyadv, nanomanufacturing, nanomaterials, ncrna, network, neuroglia, neurolint, neurosci, nitrogen, notspecified, nri, nursrep, nutraceuticals, nutrients, obesities, oceans, ohbm, onco, oncopathology, optics, oral, organics, organoids, osteology, oxygen, parasites, parasitologia, particles, pathogens, pathophysiology, pediatrrep, pharmaceuticals, pharmaceutics, pharmacoepidemiology, pharmacy, philosophies, photochem, photonics, phycology, physchem, physics, physiologia, plants, plasma, pollutants, polymers, polysaccharides, poultry, powders, preprints, proceedings, processes, prosthesis, proteomes, psf, psych, psychiatryint, psychoactives, publications, quantumrep, quaternary, qubs, radiation, reactions, recycling, regeneration, religions, remotesensing, reports, reprodmed, resources, rheumato, risks, robotics, ruminants, safety, sci, scipharm, seeds, sensors, separations, sexes, signals, sinusitis, skins, smartcities, sna, societies, socsci, software, soilsystems, solar, solids, sports, standards, stats, stresses, surfaces, surgeries, suschem, sustainability, symmetry, synbio, systems, taxonomy, technologies, telecom, test, textiles, thalassrep, thermo, tomography, tourismhosp, toxics, toxins, transplantology, transportation, traumacare, traumas, tropicalmed, universe, urbansci, uro, vaccines, vehicles, venereology, vetsci, vibration, viruses, vision, waste, water, wem, wevj, wind, women, world, youth, zoonoticdis 

%---------
% article
%---------
% The default type of manuscript is "article", but can be replaced by: 
% abstract, addendum, article, book, bookreview, briefreport, casereport, comment, commentary, communication, conferenceproceedings, correction, conferencereport, entry, expressionofconcern, extendedabstract, datadescriptor, editorial, essay, erratum, hypothesis, interestingimage, obituary, opinion, projectreport, reply, retraction, review, perspective, protocol, shortnote, studyprotocol, systematicreview, supfile, technicalnote, viewpoint, guidelines, registeredreport, tutorial
% supfile = supplementary materials

%----------
% submit
%----------
% The class option "submit" will be changed to "accept" by the Editorial Office when the paper is accepted. This will only make changes to the frontpage (e.g., the logo of the journal will get visible), the headings, and the copyright information. Also, line numbering will be removed. Journal info and pagination for accepted papers will also be assigned by the Editorial Office.

%------------------
% moreauthors
%------------------
% If there is only one author the class option oneauthor should be used. Otherwise use the class option moreauthors.

%---------
% pdftex
%---------
% The option pdftex is for use with pdfLaTeX. If eps figures are used, remove the option pdftex and use LaTeX and dvi2pdf.

%=================================================================
% MDPI internal commands
\firstpage{1} 
 
\setcounter{page}{\@firstpage} 

\pubvolume{1}
\issuenum{1}
\articlenumber{0}
\pubyear{2022}
\copyrightyear{2022}
%\externaleditor{Academic Editor: Firstname Lastname} % For journal Automation, please change Academic Editor to "Communicated by"
\datereceived{} 
\dateaccepted{} 
\datepublished{} 
%\datecorrected{} % Corrected papers include a "Corrected: XXX" date in the original paper.
%\dateretracted{} % Corrected papers include a "Retracted: XXX" date in the original paper.
\hreflink{https://doi.org/} % If needed use \linebreak
%\doinum{}
%------------------------------------------------------------------
% The following line should be uncommented if the LaTeX file is uploaded to arXiv.org
%\pdfoutput=1

%=================================================================
% Add packages and commands here. The following packages are loaded in our class file: fontenc, inputenc, calc, indentfirst, fancyhdr, graphicx, epstopdf, lastpage, ifthen, lineno, float, amsmath, setspace, enumitem, mathpazo, booktabs, titlesec, etoolbox, tabto, xcolor, soul, multirow, microtype, tikz, totcount, changepage, attrib, upgreek, cleveref, amsthm, hyphenat, natbib, hyperref, footmisc, url, geometry, newfloat, caption

%=================================================================
%% Please use the following mathematics environments: Theorem, Lemma, Corollary, Proposition, Characterization, Property, Problem, Example, ExamplesandDefinitions, Hypothesis, Remark, Definition, Notation, Assumption
%% For proofs, please use the proof environment (the amsthm package is loaded by the MDPI class).

%=================================================================
% The fields PACS, MSC, and JEL may be left empty or commented out if not applicable
%\PACS{J0101}
%\MSC{}
%\JEL{}

%%%%%%%%%%%%%%%%%%%%%%%%%%%%%%%%%%%%%%%%%%
% Only for the journal Diversity
%\LSID{\url{http://}}

%%%%%%%%%%%%%%%%%%%%%%%%%%%%%%%%%%%%%%%%%%
% Only for the journal Applied Sciences:
%\featuredapplication{Authors are encouraged to provide a concise description of the specific application or a potential application of the work. This section is not mandatory.}
%%%%%%%%%%%%%%%%%%%%%%%%%%%%%%%%%%%%%%%%%%

%%%%%%%%%%%%%%%%%%%%%%%%%%%%%%%%%%%%%%%%%%
% Only for the journal Data:
%\dataset{DOI number or link to the deposited data set in cases where the data set is published or set to be published separately. If the data set is submitted and will be published as a supplement to this paper in the journal Data, this field will be filled by the editors of the journal. In this case, please make sure to submit the data set as a supplement when entering your manuscript into our manuscript editorial system.}

%\datasetlicense{license under which the data set is made available (CC0, CC-BY, CC-BY-SA, CC-BY-NC, etc.)}

%%%%%%%%%%%%%%%%%%%%%%%%%%%%%%%%%%%%%%%%%%
% Only for the journal Toxins
%\keycontribution{The breakthroughs or highlights of the manuscript. Authors can write one or two sentences to describe the most important part of the paper.}

%%%%%%%%%%%%%%%%%%%%%%%%%%%%%%%%%%%%%%%%%%
% Only for the journal Encyclopedia
%\encyclopediadef{Instead of the abstract}
%\entrylink{The Link to this entry published on the encyclopedia platform.}
%%%%%%%%%%%%%%%%%%%%%%%%%%%%%%%%%%%%%%%%%%

\makeatother

\begin{document}
\maketitle

\section{Introduction}

A multitude of everyday problems from various sciences can be treated
as a problem of categorization or regression problems, such as problems
that appear in the fields of physics \citep{fc_physics1,nnphysics1,nnphysics2,nnphysics3},
chemistry \citep{fc_chem1,fc_chem2,fc_chem3}, economics \citep{fc_econ1,fc_econ2},
environmental problems \citep{fc_pollution1,fc_pollution2,fc_pollution3},
medical problems \citep{nnmed1,nnmed2} etc. In the relevant literature
there is a wide range of techniques that one can use to handle such
problems such as the k nearest neighboors model (k-NN) \citep{knn1,knn2},
artificial neural networks (ANNs) \citep{nn1,nn2}, Radial Basis Function
(RBF) networks \citep{rbf2,rbf3,rbf4}, Support Vector Machines (SVM)
\citep{svm,svm2}, decision trees \citep{dt1,dt2} etc. A systematic
review of methods used in classification problems can be found in
the paper of Kotsiantis et al \citep{class_review}. 

In most cases, learning models have a number of parameters that should
be determined through some algorithms, such as the Back Propagation
method \citep{bpnn,bpnn2} for the artificial neural networks or more
advanced optimization methods such as the Genetic algorithms \citep{geneticnn1,geneticnn2,geneticnn3}.
However, most of the time there are two main problems in the parameterization
of learning models:
\begin{itemize}
\item Requirement for large training time, which is proportional to the
dimension of the input data. For example, in a neural network with
one hidden layer equipped with 10 processing nodes and a provided
dataset with 10 inputs, then more than $N=100$ parameters are required
to build the neural network. Therefore the size of the network will
grow proportionally to the problem and therefore longer training times
will be required for the model. In addition, in some techniques such
as for example Bfgs \citep{Powell}, $O\left(N^{2}\right)$ storage
space will be required for the training model and for the partial
derivatives required by the optimization method. An extensive discussion
of the problems caused by increased data dimensionality is presented
in the paper by Verleysen et al \citep{nndimension}. Some common
approaches to reduce the dimension of the input datasets are the Principal
Component Analysis (PCA) method \citep{nnpca1,nnpca2,nnpca3} as well
as the Minimum redundancy feature selection (MRMR) technique \citep{mrmr1,mrmr2}.
Furthermore, Wang et al proposed an auto - encoder reduction method,
applied on a seires of large datasets\citep{nn_autoencoder}. 
\item The problem of reduced performance of models on unknown data, also
known as overfitting problem. The paper of Geman et al \citep{nngeman}
as well the article of Hawkins \citep{nnhawkins} thoroughly discuss
on the topic of overfitting. Examples of techniques proposed to overcome
this problem are the weight sharing methods \citep{nnsharing1,nnsharing2},
the prunning methods \citep{nnprunning1,nnprunning2,nnprunning3},
weight elimination \citep{nnelim1,nnelim2,nnelim3}, weight decaying
methods \citep{nndecay1,nndecay2} etc.
\end{itemize}
This article proposes a two phase method to overcome the above problems.
During the first phase, a limited number of artificial features are
created from the original ones using a method based on the Grammatical
Evolution procedure \citep{ge1}. Grammatical Evolution is an evolutionary
process, where the chromosomes are production rules of the target
BNF grammar and it has been used sucesfully in a variety of applications,
such as music composition \citep{geApp1}, economics \citep{geApp2},
symbolic regression \citep{geApp3}, robotics \citep{geApp4}, caching
algorithms \citep{geApp5} etc. These features are iteratively adjusted
using a hybrid technique based on particle swarm optimization (PSO)\citep{pso_major,pso1,pso2},
where the generated features are constrained using penalty factors
to be within a pre-defined value interval and their evaluation is
done using an RBF network. The RBF network was preferred over other
machine learning models to be used to evaluate the generated features
due to its training speed. The PSO method has been selected as the
optimization method due to its simplicity and the small number of
parameters that should be set. Also, the PSO method has been used
in many difficult problems from all areas of the sciences, such as
problems that arise in physics\textbf{ }\citep{psophysics1,psophysics2},
chemistry \citep{psochem1,psochem2}, medicine \citep{psomed1,psomed2},
economics \citep{psoecon} etc. Furthermore, the PSO method was successfully
applied recently in many practical problems such as flow shop scheduling
\citep{psoApp1}, successful development of electric vehicle charging
strategies \citep{psoApp2}, emotion recognition \citep{psoApp3},
robotics \citep{psoApp4} etc.

The idea of creating artificial features using Grammatical Evolution
was firstly introduced in the paper of Gavrilis et al\textbf{ }\citep{fc1}
and it has been succesfully applied on a series of problems, such
as Spam Identification \citep{fc2}, Fetal heart classification \citep{fc3},
epileptic oscillations\citep{fc4}, construction of Covid-19 predictive
models \citep{fc5}, performance and early drop prediction for higher
education students \citep{fc6} etc. 

Feature selection using neural networks has been also proposed in
a series of papers, such as the work of Verikas and Bacauskiene \citep{fsel1}
or the work of Kabir et al \citep{fsel2}. Moreover, Devi utlized
a Simulated Annealing approach \citep{genfc_anneal} to select the
most important features for classification datasets. Also, Neshatian
et al \citep{genfc2} developed a genetic algorithm that produces
features using an entropy based fitness function.

The rest of this article is divided as follows: in section \ref{sec:The-proposed-method}
the steps of the proposed method are fully described, in section \ref{sec:Experiments}
the used experimental datasets as well as the results obtained by
the incorporation of the proposed method are outlined and finally
in section \ref{sec:Conclusions} some conclusions are listed.

\section{The proposed method\label{sec:The-proposed-method}}

The basic steps by which the Grammatical Evolution technique produces
the artificial features are then analyzed as well as the steps of
the overall process of creating and evaluating the artificial features.

\subsection{The technique of Grammatical Evolution }

The process of Grammatical Evolution uses chromosomes that represent
production rules of the underlying BNF (Backus--Naur form) grammar\citep{bnf1}
of the objective problem. BNF grammars have been widely used to describe
the syntax of programming languages. Any BNF grammar is a set \textbf{$G=\left(N,T,S,P\right)$},
where
\begin{itemize}
\item The set \textbf{$N$ }represents the non-terminal symbols of the grammar.
Any non - terminal symbol is analyzed to a series of terminal symbols
using the production rules of the grammar.
\item \textbf{$T$ }is the set of terminal symbols.\textbf{ }
\item The non - terminal symbol $S$ represents the start symbol of the
grammar.
\item The set \textbf{$P$ }contains the production rules of the grammar.
Typically, any production rule is expressed in the form \textbf{$A\rightarrow a$
}or\textbf{ $A\rightarrow aB,\ A,B\in N,\ a\in T$.}
\end{itemize}
The process that creates a valid program, starts from the symbol $S$
and gradually replaces non-terminal symbols with the right hand of
the selected production rule from the provided chromosome. The rule
is selected with the following steps:
\begin{itemize}
\item Get the next element from the chromosome and denote it as V. 
\item Select the production rule with the scheme Rule = V mod $N_{R}$,
where $N_{R}$ is the total number of production rules for the current
non -- terminal symbol. 
\end{itemize}
The BNF grammar for the proposed method is shown in Figure \ref{fig:BNF-grammar-of}.
The constant N is the dimension of the input dataset.
\begin{figure}[H]
\caption{BNF grammar of the proposed method.\label{fig:BNF-grammar-of}}

\begin{lyxcode}
S::=\textless expr\textgreater ~~~(0)~

\textless expr\textgreater ~::=~~(\textless expr\textgreater ~\textless op\textgreater ~\textless expr\textgreater )~~(0)~~~~~~~~~~~~~

~~~~~~~~~~~\textbar ~\textless func\textgreater ~(~\textless expr\textgreater ~)~~~~(1)~~~~~~~~~~~~~

~~~~~~~~~~~\textbar\textless terminal\textgreater ~~~~~~~~~~~~(2)~

\textless op\textgreater ~::=~~~~~+~~~~~~(0)~~~~~~~~~~~~~

~~~~~~~~~~~\textbar ~-~~~~~~(1)~~~~~~~~~~~~~

~~~~~~~~~~~\textbar ~{*}~~~~~~(2)~~~~~~~~~~~~~

~~~~~~~~~~~\textbar ~/~~~~~~(3)

\textless func\textgreater ~::=~~~sin~~(0)~~~~~~~~~~~~~

~~~~~~~~~~~\textbar ~cos~~(1)~~~~~~~~~~~~~

~~~~~~~~~~~\textbar exp~~~(2)~~~~~~~~~~~~~

~~~~~~~~~~~\textbar log~~~(3)

\textless terminal\textgreater ::=\textless xlist\textgreater ~~~~~~~~~~~~~~~~(0)~~~~~~~~~~~~~~~~~~~~~~

~~~~~~~~~~~\textbar\textless digitlist\textgreater .\textless digitlist\textgreater ~(1)

\textless xlist\textgreater ::=x1~~~~(0)~~~~~~~~~~~~~~

~~~~~~~~~~~\textbar ~x2~(1)~~~~~~~~~~~~~~

~~~~~~~~~~~………~~~~~~~~~~~~~

~~~~~~~~~~~\textbar ~xN~(N)

\textless digitlist\textgreater ::=\textless digit\textgreater ~~~~~~~~~~~~~~~~~~(0)~~~~~~~~~~~~~~~~~

~~~~~~~~~~~\textbar ~\textless digit\textgreater\textless digit\textgreater ~~~~~~~~~~~~(1)

~~~~~~~~~~~\textbar ~\textless digit\textgreater\textless digit\textgreater\textless digit\textgreater ~~~~~(2)

\textless digit\textgreater ~~::=~0~(0)~~~~~~~~~~~~~

~~~~~~~~~~~\textbar ~1~(1)~~~~~~~~~~~~~

~~~~~~~~~~~\textbar ~2~(2)~~~~~~~~~~~~~

~~~~~~~~~~~\textbar ~3~(3)~~~~~~~~~~~~~

~~~~~~~~~~~\textbar ~4~(4)~~~~~~~~~~~~~

~~~~~~~~~~~\textbar ~5~(5)~~~~~~~~~~~~~

~~~~~~~~~~~\textbar ~6~(6)~~~~~~~~~~~~~

~~~~~~~~~~~\textbar ~7~(7)~~~~~~~~~~~~~

~~~~~~~~~~~\textbar ~8~(8)~~~~~~~~~~~~~

~~~~~~~~~~~\textbar ~9~(9)
\end{lyxcode}
\end{figure}
 An example that produces a valid expression for the chromosome 
\[
x=\left[9,8,6,4,16,10,17,23,8,14\right]
\]
with $N=3$ is shown in Table \ref{tab:table_with_steps}. The final
expression that created is \textbf{$f(x)=x_{2}+\cos\left(x_{3}\right)$.}

\begin{table}[H]
\caption{Steps to produce a valid expression from the BNF grammar.\label{tab:table_with_steps}}

\begin{tabular}{|c|c|c|}
\hline 
Expression & Chromosome & Operation\tabularnewline
\hline 
\hline 
\textless expr\textgreater{} & 9,8,6,4,16,10,17,23,8,14 & $9\mod3=0$\tabularnewline
\hline 
(\textless expr\textgreater\textless op\textgreater\textless expr\textgreater ) & 8,6,4,16,10,17,23,8,14 & $8\mod3=2$\tabularnewline
\hline 
(\textless terminal\textgreater\textless op\textgreater\textless expr\textgreater ) & 6,4,16,10,17,23,8,14 & $6\mod2=0$\tabularnewline
\hline 
(\textless xlist\textgreater\textless op\textgreater\textless expr\textgreater ) & 4,16,10,17,23,8,14 & $4\mod3=1$\tabularnewline
\hline 
(x2\textless op\textgreater\textless expr\textgreater ) & 16,10,17,23,8,14 & $16\mod4=0$\tabularnewline
\hline 
(x2+\textless expr\textgreater ) & 10,17,23,8,14 & $10\mod3=1$\tabularnewline
\hline 
(x2+\textless func\textgreater (\textless expr\textgreater )) & 17,23,8,14 & $17\mod4=1$\tabularnewline
\hline 
(x2+cos(\textless expr\textgreater )) & 23,8,14 & $23\mod2=1$\tabularnewline
\hline 
(x2+cos(\textless terminal\textgreater )) & 8,14 & $8\mod2=0$\tabularnewline
\hline 
(x2+cos(\textless xlist\textgreater )) & 14 & $14\mod3=2$\tabularnewline
\hline 
(x2+cos(x3)) &  & \tabularnewline
\hline 
\end{tabular}
\end{table}


\subsection{Feature construction\label{subsec:Feature-construction}}

The proposed method is used to created $N_{f}$ artificial features
from the original ones. The new features will be considered as non
- linear combinations of the old features and the process for any
particle $p$ has as follows:
\begin{enumerate}
\item \textbf{Divide} $p$ into $N_{f}$ parts. Every part is denoted as
the $p_{i}$ sub - particle.
\item \textbf{For} each sub - particle $p_{i}$ a new artificial feature
$g_{i}\left(\overrightarrow{x},p_{i}\right)$ is constructed with
the grammar of Figure \ref{fig:BNF-grammar-of} as a non - linear
combination of the original set of features $\overrightarrow{x}$.
\end{enumerate}
The final set of features will be considered as mapping functions
of the original ones. For example the set:

\[
g(\overrightarrow{x},p)=\begin{cases}
\begin{array}{ccc}
g_{1}(\overrightarrow{x},p_{1}) & = & x_{1}^{2}+2x_{3}\\
g_{2}(\overrightarrow{x},p_{2}) & = & 3\cos\left(x_{2}\right)
\end{array}\end{cases}
\]
is a set of mapping functions for the original features $\overrightarrow{x}=\left(x_{1},x_{2},x_{3}\right)$.
However, sometimes the generated features can lead to extreme values
and this will result in generalization problems from the used machine
learning models. For this reason and in the present work, penalty
factors are used so that the mapping functions do not lead to extreme
values.\textbf{ }These penalty factors also modify the fitness function
that the Particle Swarm Optimization technique will minimize each
time and are considered next.

\subsection{Fitness calculation\label{subsec:Fitness-calculation}}

The following steps calculates the fitness for any given particle
$p$.
\begin{enumerate}
\item \textbf{Denote as} $\mbox{TO}=\left\{ \left(\overrightarrow{x_{1}},y_{1}\right),\left(\overrightarrow{x_{2}},y_{2}\right),\ldots,\left(\overrightarrow{x_{M}},y_{M}\right)\right\} $
the original train set.
\item \textbf{Set} $V=0$, the penalty factor
\item \textbf{Compute} the mapping function $g(\overrightarrow{x},p)$ as
suggested in subsection \ref{subsec:Feature-construction}
\item \textbf{Set} $\mbox{TF=\ensuremath{\emptyset}},$the modified train
set
\item \textbf{For} $i=1,\ldots,M$ \textbf{do}
\begin{enumerate}
\item \textbf{Set} $\widetilde{x_{i}}=g\left(\overrightarrow{x_{i}},p\right)$
\item \textbf{Set} $\mbox{TF}=\mbox{TF}\cup\left(\widetilde{x_{i}},y_{i}\right)$
\item \textbf{If} $\left\Vert \widetilde{x_{i}}\right\Vert >L_{\mbox{max}}$,
\textbf{then} $V=V+1$, where $L_{\mbox{max}}$ a predefined positive
value. 
\end{enumerate}
\item \textbf{End For }
\item \textbf{Train} an RBF $C(x)$ with $H$ processing NODES on $\mbox{TF}$
and obtain the following error:
\begin{equation}
f_{p}=\sum_{j=1}^{M}\left(C\left(\widetilde{x_{i}}\right)-y_{j}\right)^{2}
\end{equation}
\item \textbf{Compute} the final fitness value: 
\begin{equation}
f_{p}=f_{p}\times\left(1+\lambda V^{2}\right)
\end{equation}
where $\lambda>0$.
\end{enumerate}

\subsection{The proposed PSO method}

The mains steps for this algorithm are outlined in detail in Algorithm
\ref{alg:psoSerial}.

\begin{algorithm}[H]
\caption{The base PSO algorithm executed in one processing unit.\label{alg:psoSerial}}

\begin{enumerate}
\item \textbf{Initialization Step} . 

\begin{enumerate}
\item \textbf{Set} $\text{\mbox{iter}}=0$.
\item \textbf{Set} $m$ as the total number of particles.
\item \textbf{Set }$\mbox{iter}_{\mbox{max}}$ as the maximum number of
iterations allowed.
\item \textbf{Initialize} randomly, the positions $p_{1},p_{2},...,p_{m}$
for the particles. For the grammatical evolution, every chromosome
is a series of randomly selected integers.
\item \textbf{Initialize} randomly the velocities $u_{1},u_{2},...,u_{m}$.
For the current work every vector of velocities is a series of randomly
selected integers in a the range $\left[u_{\mbox{min}},u_{\mbox{max}}\right]$.
In the current work $u_{\mbox{min}}=-5,\ u_{\mbox{max}}=5$.
\item \textbf{For} $i=1..m$ do $b_{i}=p_{i}$. The vector $b_{i}$ denotes
the best located position of particle $p_{i}$.
\item \textbf{Set} $p_{\mbox{best}}=\arg\min_{i\in1..m}f\left(p_{i}\right)$
\end{enumerate}
\item \textbf{Termination Check Step} .\label{enu:Check-Termination.} If
$\mbox{iter}\ge\mbox{iter}_{\mbox{max}}$ then goto step \ref{enu:Termination-step.}.
\item \textbf{For} $i=1..m$ \textbf{Do\label{enu:For}}

\begin{enumerate}
\item \textbf{Compute }the velocity $u_{i}$ as a combination of the vectors
$u_{i},\ p_{i}$ and $p_{\mbox{best}}$
\item \textbf{Set} the new position for the particle as: $p_{i}=p_{i}+u_{i}$\label{enu:Update-the-position}
\item \textbf{Calculate} the fitness $f\left(p_{i}\right)$ for particle
$p_{i}$ using the procedure described in subsection \ref{subsec:Fitness-calculation}.
\item \textbf{If} $f\left(p_{i}\right)\le f\left(b_{i}\right)$ then $b_{i}=p_{i}$
\end{enumerate}
\item \textbf{End} \textbf{For}
\item \textbf{Set} $p_{\mbox{best}}=\arg\min_{i\in1..m}f\left(p_{i}\right)$
\item \textbf{Set} $\mbox{iter}=\mbox{iter}+1$. \label{enu:update_k}
\item \textbf{Goto} Step \ref{enu:Check-Termination.}
\item \textbf{Test step}.\label{enu:Termination-step.} Apply the mapping
function of the best particle $p_{\mbox{best}}$ to the test set of
the problem and apply a machine learning model obtaining the corresponding
test error.
\end{enumerate}
\end{algorithm}
The above calculates at every iteration the new position of the particle
$i$ using:
\begin{equation}
p_{i}=p_{i}+u_{i}\label{eq:eq3}
\end{equation}
In most cases the new velocity could be a linear combination of the
previously computed velocity and the best values $b_{i}$ and $p_{\mbox{best}}$
and it can be defined as:
\begin{equation}
u_{i}=\omega u_{i}+r_{1}c_{1}\left(b_{i}-p_{i}\right)+r_{2}c_{2}\left(p_{\mbox{best}}-p_{i}\right)\label{eq:eq4-1}
\end{equation}
where 
\begin{enumerate}
\item The variables $r_{1},\ r_{2}$ are random numbers defined in $[0,1].$
\item The constants $c_{1},\ c_{2}$ are defined in range $[1,2]$. 
\item The variable $\omega$, commnly called inertia, was suggested by Shi
and Eberhart \citep{pso_major} and in the current work was computer
through the following equation
\end{enumerate}
\begin{equation}
\omega_{\mbox{iter}}=0.5+\frac{r}{2}
\end{equation}
The variable $r$ is a a random number with $r\in[0,1]$.

\section{Experiments \label{sec:Experiments}}

The ability of the proposed technique to produce effective artificial
features for class prediction and feature learning will be measured
in this section on a range of datasets from the relevant literature.
This data comes from the relevant websites:
\begin{enumerate}
\item UCI dataset repository, \url{https://archive.ics.uci.edu/ml/index.php}
\item Keel repository, \url{https://sci2s.ugr.es/keel/datasets.php}\citep{Keel}.
\item The Statlib URL \url{ftp://lib.stat.cmu.edu/datasets/index.html }. 
\end{enumerate}
The proposed technique will be compared with a series of known machine
learning techniques and the experimental results are then presented
in the relevant tables.

\subsection{Experimental datasets }

The classification problems used in the experiments have as follows:
\begin{enumerate}
\item \textbf{Appendictis}, a medical dataset that represents 7 medical
measures for 106 patients on which the class label represents if the
patient has appendicitis \citep{appendicitis,appendicitis2}. 
\item \textbf{Australian} dataset \citep{australian}, an dataset concerning
economical transactions in banks.
\item \textbf{Balance} dataset, a dataset generated to model psychological
experimental results\citep{balance}.
\item \textbf{Bands} dataset, a dataset used in rotogravure printing \citep{bands}.
\item \textbf{Dermatology} dataset \citep{dermatology}, a medical dataset
used to detect the type of Eryhemato-Squamous Disease. 
\item \textbf{Hayes roth} dataset \citep{hayesroth}.
\item \textbf{Heart} dataset \citep{heart}, a mecical dataset used to detect
heart diseases. 
\item \textbf{HouseVotes} dataset \citep{housevotes}, a dataset related
to the conressional voting records of USA.
\item \textbf{Ionosphere} dataset, used to classify measurements from the
ionosphere and it has been examined in a variety of research papers
\citep{ion1,ion2}.
\item \textbf{Liverdisorder} dataset \citep{liver1,liver2}, a medical dataset.
\item \textbf{Mammographic} dataset \citep{mammographic}, a medical dataset
used for breast cancer diagnosis.
\item \textbf{Parkinsons} dataset \citep{parkinsons1,parkinsons2}, a dataset
used to detect the Parkinson's decease using voice measurements.
\item \textbf{Pima} dataset \citep{pima}, a medical dataset.
\item \textbf{Popfailures} dataset \citep{popfailures}, a dataset related
to meterological data.
\item \textbf{Regions2} dataset, a medical dataset produced by liver biopsy
images of patients with hepatitis C \citep{regions}. 
\item \textbf{Saheart} dataset \citep{saheart}, a medical dataset. 
\item \textbf{Segment} dataset \citep{segment}, a dataset related to image
segmentation.
\item \textbf{Wdbc} dataset \citep{wdbc}, a dataset used to detect breast
tumors. 
\item \textbf{Wine} dataset, a dataset related to chemical analysis of wines
\citep{wine1,wine2}.
\item \textbf{Eeg} datasets \citep{eeg1,eeg2}, it is medical datasets about
EEG signals and the following cases were used in the experiments: 
\begin{enumerate}
\item Z\_F\_S, 
\item ZO\_NF\_S
\item ZONF\_S.
\end{enumerate}
\item \textbf{Zoo} dataset \citep{zoo}.
\end{enumerate}
The regression datasets used in the relevant experiments have as follows:
\begin{enumerate}
\item \textbf{Abalone} dataset \citep{abalone}, a dataset used to predict
the age of abalones.
\item \textbf{Airfoil }dataset, a dataset provided by NASA \citep{airfoil}
obtained from a series of aerodynamic and acoustic tests.
\item \textbf{Baseball} dataset, a dataset related to the salary of baseball
players.
\item \textbf{BK} dataset \citep{Stat}, a dataset that was used to calculate
the points in a basketball game. 
\item \textbf{BL} dataset, used in machine problems.
\item \textbf{Concrete} dataset \citep{concrete}, a civil engineering dataset
to calculate The concrete compressive strength
\item \textbf{Dee} dataset, used to estimate the daily average price of
TkWhe electricity energy in Spain.
\item \textbf{Diabetes} dataset, a medical dataset.
\item \textbf{Housing} dataset \citep{key23}.
\item \textbf{FA} dataset, used to fit body fat to other measurements. 
\item \textbf{MB} dataset \citep{key21}. 
\item \textbf{MORTGAGE} dataset, holding economic data from USA. The goal
is to predict the 30-Year Conventional Mortgage Rate. 
\item \textbf{PY }dataset, (Pyrimidines problem)\citep{pydataset}.
\item \textbf{Quake} dataset, used to approximate the strength of a earthquake
given its the depth of its focal point, its latitude and its longitude. 
\item \textbf{Treasure} dataset, which contains Economic data information
of USA, where the the goal is to predict 1-Month CD Rate. 
\end{enumerate}

\subsection{Experimental results}

For greater reliability in the experimental results the method of
10 - fold cross validation was incorporated for every experimental
dataset. Every experiment was repeated 30 times using different seed
for the random generator each time. All the used code was implemented
in ANSI C++ using the OPTIMUS programming library for optimization
purposes, freely available from \url{https://github.com/itsoulos/OPTIMUS/}.
For the classification datasets the average classification error as
measured in the test set is reported while for the regression datasets
the average regression error is reported. Also, in every table and
additional column denoted as AVERAGE is added to show the average
classification or regression error for the corresponding datasets.
The values for the experimental parameters are shown in Table \ref{tab:values}.

\begin{table}[H]

\caption{The values for every parameter used in the experimens.\label{tab:values}}

\centering{}%
\begin{tabular}{|c|c|c|}
\hline 
PARAMETER & MEANING & VALUE\tabularnewline
\hline 
\hline 
$m$ & Particles or Chromosomes & 200\tabularnewline
\hline 
$H$ & Number of hidden nodes  & 10\tabularnewline
\hline 
$\mbox{iter}_{\mbox{max}}$ & Maximum number of iterations & 200\tabularnewline
\hline 
$L_{\mbox{max}}$ & Limit used in penalty calculation & 100\tabularnewline
\hline 
$\lambda$ & Penalty factor & 100\tabularnewline
\hline 
\end{tabular}
\end{table}

The proposed technique that created artificial features is compared
on the same datasets against a series of well - known method from
the relevant literature:
\begin{enumerate}
\item A genetic algorithm with $m$ chromosomes, denotes as GENETIC in the
experimental tables. This genetic algorithm is used to train an artificial
neural network with $H$ hidden nodes. After the termination of the
genetic algorithm the local optimization method BFGS is applied to
the best chromosome of the population.
\item The Radial Basis Function (RBF) network \citep{rbf1} with $H$ processing
nodes.
\item The optimization method Adam \citep{Adam}, used to train an artificial
neural netwok with $H$ hidden nodes.
\item The Rprop optimization method \citep{rpropnn,rpropnn2,rpropnn3},
used to train an artificial neural network with $H$ hidden nodes.
\item The NEAT method (NeuroEvolution of Augmenting Topologies ) \citep{neat}.
\end{enumerate}
The experimental results using the above methods on the classification
datasets are shown in Table \ref{tab:exp1} and the results for the
regression datasets are illustrated in Table \ref{tab:exp2}.

\begin{table}[H]
\caption{Average classification error for the classification datasets using
the well - known methods. \label{tab:exp1}}

\centering{}%
\begin{tabular}{|c|c|c|c|c|c|}
\hline 
DATASET & GENETIC & RBF & ADAM & RPROP & NEAT\tabularnewline
\hline 
\hline 
Appendicitis & 18.10\% & 12.23\% & 16.50\% & 16.30\% & 17.20\%\tabularnewline
\hline 
Australian & 32.21\% & 34.89\% & 35.65\% & 36.12\% & 31.98\%\tabularnewline
\hline 
Balance & 8.97\% & 33.42\% & 7.87\% & 8.81\% & 23.14\%\tabularnewline
\hline 
Bands & 35.75\% & 37.22\% & 36.25\% & 36.32\% & 34.30\%\tabularnewline
\hline 
Dermatology & 30.58\% & 62.34\% & 26.14\% & 15.12\% & 32.43\%\tabularnewline
\hline 
Hayes Roth & 56.18\% & 64.36\% & 59.70\% & 37.46\% & 50.15\%\tabularnewline
\hline 
Heart & 28.34\% & 31.20\% & 38.53\% & 30.51\% & 39.27\%\tabularnewline
\hline 
HouseVotes & 6.62\% & 6.13\% & 7.48\% & 6.04\% & 10.89\%\tabularnewline
\hline 
Ionosphere & 15.14\% & 16.22\% & 16.64\% & 13.65\% & 19.67\%\tabularnewline
\hline 
Liverdisorder & 31.11\% & 30.84\% & 41.53\% & 40.26\% & 30.67\%\tabularnewline
\hline 
Lymography & 23.26\% & 25.31\% & 29.26\% & 24.67\% & 33.70\%\tabularnewline
\hline 
Mammographic & 19.88\% & 21.38\% & 46.25\% & 18.46\% & 22.85\%\tabularnewline
\hline 
Parkinsons & 18.05\% & 17.42\% & 24.06\% & 22.28\% & 18.56\%\tabularnewline
\hline 
Pima & 32.19\% & 25.78\% & 34.85\% & 34.27\% & 34.51\%\tabularnewline
\hline 
Popfailures & 5.94\% & 7.04\% & 5.18\% & 4.81\% & 7.05\%\tabularnewline
\hline 
Regions2 & 29.39\% & 38.29\% & 29.85\% & 27.53\% & 33.23\%\tabularnewline
\hline 
Saheart & 34.86\% & 32.19\% & 34.04\% & 34.90\% & 34.51\%\tabularnewline
\hline 
Segment & 57.72\% & 59.68\% & 49.75\% & 52.14\% & 66.72\%\tabularnewline
\hline 
Wdbc & 8.56\% & 7.27\% & 35.35\% & 21.57\% & 12.88\%\tabularnewline
\hline 
Wine & 19.20\% & 31.41\% & 29.40\% & 30.73\% & 25.43\%\tabularnewline
\hline 
Z\_F\_S & 10.73\% & 13.16\% & 47.81\% & 29.28\% & 38.41\%\tabularnewline
\hline 
ZO\_NF\_S & 8.41\% & 9.02\% & 47.43\% & 6.43\% & 43.75\%\tabularnewline
\hline 
ZONF\_S & 2.60\% & 4.03\% & 11.99\% & 27.27\% & 5.44\%\tabularnewline
\hline 
ZOO & 16.67\% & 21.93\% & 14.13\% & 15.47\% & 20.27\%\tabularnewline
\hline 
\textbf{AVERAGE} & \textbf{22.94\%} & \textbf{26.78\%} & \textbf{30.24\%} & \textbf{24.60\%} & \textbf{28.63\%}\tabularnewline
\hline 
\end{tabular}
\end{table}

\begin{table}[H]
\caption{Average regression error using the well - known methods for the regression
datasets.\label{tab:exp2}}

\centering{}%
\begin{tabular}{|c|c|c|c|c|c|}
\hline 
DATASET & GENETIC & RBF & ADAM & RPROP & NEAT\tabularnewline
\hline 
\hline 
ABALONE & 7.17 & 7.37 & 4.30 & 4.55 & 9.88\tabularnewline
\hline 
AIRFOIL & 0.003 & 0.27 & 0.005 & 0.002 & 0.067\tabularnewline
\hline 
BASEBALL & 103.60 & 93.02 & 77.90 & 92.05 & 100.39\tabularnewline
\hline 
BK & 0.027 & 0.02 & 0.03 & 1.599 & 0.15\tabularnewline
\hline 
BL & 5.74 & 0.01 & 0.28 & 4.38 & 0.05\tabularnewline
\hline 
CONCRETE & 0.0099 & 0.011 & 0.078 & 0.0086 & 0.081\tabularnewline
\hline 
DEE & 1.013 & 0.17 & 0.63 & 0.608 & 1.512\tabularnewline
\hline 
DIABETES & 19.86 & 0.49 & 3.03 & 1.11 & 4.25\tabularnewline
\hline 
HOUSING & 43.26 & 57.68 & 80.20 & 74.38 & 56.49\tabularnewline
\hline 
FA & 1.95 & 0.02 & 0.11 & 0.14 & 0.19\tabularnewline
\hline 
MB & 3.39 & 2.16 & 0.06 & 0.055 & 0.061\tabularnewline
\hline 
MORTGAGE & 2.41 & 1.45 & 9.24 & 9.19 & 14.11\tabularnewline
\hline 
PY & 1.21 & 0.02 & 0.09 & 0.039 & 0.075\tabularnewline
\hline 
QUAKE & 0.04 & 0.071 & 0.06 & 0.041 & 0.298\tabularnewline
\hline 
TREASURY & 2.929 & 2.02 & 11.16 & 10.88 & 15.52\tabularnewline
\hline 
\textbf{AVERAGE} & \textbf{12.84} & \textbf{10.30} & \textbf{11.70} & \textbf{12.44} & \textbf{12.70}\tabularnewline
\hline 
\end{tabular}
\end{table}
The results using the proposed method and for the construction of
2,3 and 4 artificial features are presented in the relevant tables
\ref{tab:tabClass} and \ref{tab:tabRegression}. The RBF column represents
the experimental results in which after the construction of the artificial
features, a RBF network with $H$ processing nodes is applied on the
modified dataset. Also, the column GENETIC in tables \ref{tab:tabClass},
\ref{tab:tabRegression} stands for the results obtained by the application
of a genetic algorithm with $m$ chromosomes to the modified dataset,
when the feature creation procedure has been finished.

\begin{table}[H]
\caption{Experimental results for the classification datasets using the proposed
method. Number in cells denote average classification error as measured
on the test set.\label{tab:tabClass}}

\centering{}%
\begin{tabular}{|c|c|c|c|c|c|c|}
\hline 
 & \multicolumn{2}{c|}{$f=2$} & \multicolumn{2}{c|}{$f=3$} & \multicolumn{2}{c|}{$f=4$}\tabularnewline
\hline 
\hline 
DATASET & RBF & GENETIC & RBF & GENETIC & RBF & GENETIC\tabularnewline
\hline 
APPENDICITIS & 15.40\% & 14.33\% & 16.90\% & 15.77\% & 15.97\% & 17.30\%\tabularnewline
\hline 
AUSTRALIAN & 15.49\% & 14.48\% & 14.53\% & 15.33\% & 14.75\% & 15.97\%\tabularnewline
\hline 
BALANCE & 16.67\% & 2.89\% & 22.54\% & 4.94\% & 17.26\% & 4.62\%\tabularnewline
\hline 
BANDS & 38.09\% & 38.13\% & 37.09\% & 39.22\% & 37.22\% & 35.51\%\tabularnewline
\hline 
DERMATOLOGY & 41.58\% & 30.37\% & 35.46\% & 25.44\% & 40.45\% & 21.97\%\tabularnewline
\hline 
HAYES ROTH & 37.41\% & 27.92\% & 38.10\% & 25.74\% & 39.59\% & 25.82\%\tabularnewline
\hline 
HEART & 21.53\% & 17.13\% & 17.64\% & 16.87\% & 19.63\% & 15.69\%\tabularnewline
\hline 
HOUSEVOTES & 6.36\% & 3.78\% & 7.17\% & 3.25\% & 4.25\% & 3.52\%\tabularnewline
\hline 
IONOSPHERE & 10.32\% & 10.17\% & 10.12\% & 10.01\% & 11.42\% & 9.02\%\tabularnewline
\hline 
LIVERDISORDER & 34.23\% & 32.33\% & 35.84\% & 32.97\% & 35.93\% & 30.74\%\tabularnewline
\hline 
LYMOGRAPHY & 34.93\% & 28.67\% & 32.00\% & 23.00\% & 29.00\% & 23.83\%\tabularnewline
\hline 
MAMMOGRAPHIC & 16.92\% & 16.51\% & 16.47\% & 16.35\% & 17.54\% & 16.50\%\tabularnewline
\hline 
PARKINSONS & 11.14\% & 13.00\% & 11.30\% & 11.11\% & 12.95\% & 9.42\%\tabularnewline
\hline 
PIMA & 22.85\% & 22.76\% & 24.93\% & 24.67\% & 24.25\% & 24.20\%\tabularnewline
\hline 
POPFAILURES & 7.32\% & 7.41\% & 6.96\% & 7.62\% & 5.96\% & 5.67\%\tabularnewline
\hline 
REGIONS2 & 28.52\% & 26.84\% & 24.91\% & 25.28\% & 25.35\% & 24.93\%\tabularnewline
\hline 
SAHEART & 29.29\% & 28.63\% & 28.92\% & 30.31\% & 28.25\% & 30.25\%\tabularnewline
\hline 
SEGMENT & 52.69\% & 45.59\% & 46.83\% & 41.06\% & 50.15\% & 39.52\%\tabularnewline
\hline 
WDBC & 5.00\% & 4.66\% & 5.76\% & 4.97\% & 5.13\% & 3.84\%\tabularnewline
\hline 
WINE & 8.92\% & 7.22\% & 6.76\% & 5.75\% & 6.00\% & 5.86\%\tabularnewline
\hline 
Z\_F\_S & 7.91\% & 8.37\% & 7.89\% & 7.67\% & 5.21\% & 6.86\%\tabularnewline
\hline 
ZO\_NF\_S & 6.90\% & 6.85\% & 6.95\% & 5.65\% & 6.24\% & 5.28\%\tabularnewline
\hline 
ZONF\_S & 3.08\% & 3.40\% & 2.44\% & 2.52\% & 3.47\% & 3.33\%\tabularnewline
\hline 
ZOO & 26.47\% & 7.83\% & 31.73\% & 10.03\% & 28.70\% & 11.57\%\tabularnewline
\hline 
\textbf{AVERAGE} & \textbf{20.79\%} & \textbf{17.47\%} & \textbf{20.39\%} & \textbf{16.90\%} & \textbf{20.19\%} & \textbf{16.30\%}\tabularnewline
\hline 
\end{tabular}
\end{table}
\begin{table}[H]
\caption{Experimental results on the regression datasets using the proposed
method. The number in cells denote average regression error as measured
on the test set.\label{tab:tabRegression}}

\centering{}%
\begin{tabular}{|c|c|c|c|c|c|c|}
\hline 
 & \multicolumn{2}{c|}{$f=2$} & \multicolumn{2}{c|}{$f=3$} & \multicolumn{2}{c|}{$f=4$}\tabularnewline
\hline 
\hline 
DATASET & RBF & GENETIC & RBF & GENETIC & RBF & GENETIC\tabularnewline
\hline 
ABALONE & 4.361 & 3.518 & 4.159 & 3.839 & 4.859 & 3.786\tabularnewline
\hline 
AIRFOIL & 0.003 & 0.001 & 0.003 & 0.001 & 0.003 & 0.001\tabularnewline
\hline 
BASEBALL & 66.00 & 53.74 & 60.79 & 57.04 & 66.19 & 61.69\tabularnewline
\hline 
BK & 0.022 & 0.031 & 0.021 & 0.029 & 0.019 & 0.023\tabularnewline
\hline 
BL & 0.413 & 0.0001 & 0.019 & 0.007 & 0.043 & 0.011\tabularnewline
\hline 
CONCRETE & 0.008 & 0.006 & 0.007 & 0.005 & 0.008 & 0.004\tabularnewline
\hline 
DEE & 0.259 & 0.252 & 0.339 & 0.286 & 0.609 & 0.5\tabularnewline
\hline 
DIABETES & 0.611 & 0.832 & 0.634 & 1.411 & 0.857 & 1.157\tabularnewline
\hline 
HOUSING & 22.387 & 15.583 & 18.614 & 13.602 & 14.83 & 13.208\tabularnewline
\hline 
FA & 0.056 & 0.011 & 0.015 & 0.011 & 0.015 & 0.012\tabularnewline
\hline 
MB & 0.258 & 0.087 & 0.115 & 0.078 & 0.342 & 0.072\tabularnewline
\hline 
MORTGAGE & 0.621 & 0.046 & 0.65 & 0.037 & 0.078 & 0.04\tabularnewline
\hline 
PY & 2.894 & 0.14 & 0.936 & 0.029 & 0.724 & 0.031\tabularnewline
\hline 
QUAKE & 0.069 & 0.036 & 0.057 & 0.037 & 0.04 & 0.037\tabularnewline
\hline 
TREASURY & 0.912 & 0.088 & 0.874 & 0.084 & 0.173 & 0.076\tabularnewline
\hline 
\textbf{AVERAGE} & \textbf{6.59} & \textbf{4.96} & \textbf{5.82} & \textbf{5.10} & \textbf{5.92} & \textbf{5.38}\tabularnewline
\hline 
\end{tabular}
\end{table}

As can be seen from the experimental results, the proposed technique
is able to significantly reduce the error in the corresponding test
sets. Especially in the case of regression problems, the reduction
in error is on average greater than 50\%. Moreover, the usage of a
neural network trained by a genetic algorithm on the modified datasets,
gives clearly better results than the use of an RBF neural network,
especially in the classification datasets.

\textbf{(Na grafei pos ta features exon ftiaxtei me RBF epeidi einai
grigori diadikasia)}

\textbf{(Na grafei pos yparxei simantiki meiosi stin diastasi ton
dedomenon eisodou)}

\textbf{(Grafimata apo alexandros)}

\section{Conclusions\label{sec:Conclusions} }

A hybrid technique that utilizes a Particle Swarm Optimizer and a
feature creation method using Grammatical Evolution was introduced
here. The proposed method can identify possible dependecies between
the original features and also can reduce the number of required features
to a limited number. Also, the method can remove from the set of features
those features that may not contribute to the learning of the data
set by some machine learning model. In addition, to make learning
more efficient the values of the generated features are bounded within
a value interval using penalty factors. The constructed features are
evaluated in terms of their effectiveness with the help of a fast
machine learning model such as the RBF network, even though other
more effective models could also be used. Among the advantages of
the proposed procedure is the fact that it does not require any prior
knowledge of the data set to which it will be applied and furthermore
the procedure is exactly the same whether it is a data classification
problem or a data fitting problem. The Particle Swarm Optimization
method was used for the production of the characteristics as it has
been proven by the relevant literature to be an extremely efficient
technique and has a limited number of parameters that must be defined
by the user.

The proposed method was applied on a extended series of widely used
datasets from the relevant literature and was compared against some
machine learning models on the same datasets. From the experimental
results, it was seen that the proposed technique dramatically improves
the performance of traditional learning techniques when applied to
artificial features. This improvement reaches an average of 30\% for
data classification and 50\% for data fitting problems. Furthermore,
as shown in the experimental results, the proposed technique is able
to give excellent results even when only two features are used. Future
extensions of the method may include the use of parallel techniques
for feature construction to drastically reduce the required execution
time.

\vspace{6pt}


\authorcontributions{I.G.T. and A.T. conceived of the idea and the methodology and I.G.T
has implemented the corresponding software. I.G.T. conducted the experiments,
employing objective functions as test cases, and provided the comparative
experiments. A.T. has performed the necessary statistical tests. All
authors have read and agreed to the published version of the manuscript.}

\funding{This research received no external funding.}

\institutionalreview{Not applicable.}

\informedconsent{Not applicable. }

\institutionalreview{Not applicable.}

\acknowledgments{The experiments of this research work were performed at the high
performance computing system established at Knowledge and Intelligent
Computing Laboratory, Department of Informatics and Telecommunications,
University of Ioannina, acquired with the project “Educational Laboratory
equipment of TEI of Epirus” with MIS 5007094 funded by the Operational
Programme “Epirus” 2014--2020, by ERDF and national funds.}

\conflictsofinterest{The authors declare no conflict of interest.}

\sampleavailability{Not applicable.}

\appendixtitles{}

\appendixstart{}

\appendix

\begin{adjustwidth}{-\extralength}{0cm}{}

\reftitle{References}
\begin{thebibliography}{99}
\bibitem{fc_physics1}E.M. Metodiev, B. Nachman, J. Thaler, Classification
without labels: learning from mixed samples in high energy physics.
J. High Energ. Phys. 2017, article number 174, 2017.

\bibitem{nnphysics1}P. Baldi, K. Cranmer, T. Faucett et al, Parameterized
neural networks for high-energy physics, Eur. Phys. J. C \textbf{76},
2016.

\bibitem{nnphysics2}J. J. Valdas and G. Bonham-Carter, Time dependent
neural network models for detecting changes of state in complex processes:
Applications in earth sciences and astronomy, Neural Networks \textbf{19},
pp. 196-207, 2006

\bibitem{nnphysics3}G. Carleo,M. Troyer, Solving the quantum many-body
problem with artificial neural networks, Science \textbf{355}, pp.
602-606, 2017.

\bibitem{fc_chem1}C. Güler, G. D. Thyne, J. E. McCray, K.A. Turner,
Evaluation of graphical and multivariate statistical methods for classification
of water chemistry data, Hydrogeology Journal \textbf{10}, pp. 455-474,
2002

\bibitem{fc_chem2}E. Byvatov ,U. Fechner ,J. Sadowski , G. Schneider,
Comparison of Support Vector Machine and Artificial Neural Network
Systems for Drug/Nondrug Classification, J. Chem. Inf. Comput. Sci.
\textbf{43}, pp 1882--1889, 2003.

\bibitem{fc_chem3}Kunwar P. Singh, Ankita Basant, Amrita Malik, Gunja
Jain, Artificial neural network modeling of the river water quality---A
case study, Ecological Modelling \textbf{220}, pp. 888-895, 2009.

\bibitem{fc_econ1}I. Kaastra, M. Boyd, Designing a neural network
for forecasting financial and economic time series, Neurocomputing
\textbf{10}, pp. 215-236, 1996. 

\bibitem{fc_econ2}Moshe Leshno, Yishay Spector, Neural network prediction
analysis: The bankruptcy case, Neurocomputing \textbf{10}, pp. 125-147,
1996.

\bibitem{fc_pollution1}A. Astel, S. Tsakovski, V, Simeonov et al.,
Multivariate classification and modeling in surface water pollution
estimation. Anal Bioanal Chem \textbf{390}, pp. 1283--1292, 2008.

\bibitem{fc_pollution2}A. Azid, H. Juahir, M.E. Toriman et al., Prediction
of the Level of Air Pollution Using Principal Component Analysis and
Artificial Neural Network Techniques: a Case Study in Malaysia, Water
Air Soil Pollut \textbf{225}, pp. 2063, 2014.

\bibitem{fc_pollution3}H. Maleki, A. Sorooshian, G. Goudarzi et al.,
Air pollution prediction by using an artificial neural network model,
Clean Techn Environ Policy \textbf{21}, pp. 1341--1352, 2019.

\bibitem{nnmed1}Igor I. Baskin, David Winkler and Igor V. Tetko,
A renaissance of neural networks in drug discovery, Expert Opinion
on Drug Discovery \textbf{11}, pp. 785-795, 2016.

\bibitem{nnmed2}Ronadl Bartzatt, Prediction of Novel Anti-Ebola Virus
Compounds Utilizing Artificial Neural Network (ANN), Chemistry Faculty
Publications \textbf{49}, pp. 16-34, 2018.

\bibitem{knn1}Y. Wu, K. Ianakiev, V. Govindaraju, Improved k-nearest
neighbor classification, Pattern Recognition \textbf{35}, pp. 2311-2318,
2002.

\bibitem{knn2}Z.-ga Liu, Q. Pan, J. Dezert, A new belief-based K-nearest
neighbor classification method, Pattern Recognition \textbf{46}, pp.
834-844, 2012.

\bibitem{nn1}C. Bishop, Neural Networks for Pattern Recognition,
Oxford University Press, 1995.

\bibitem{nn2}G. Cybenko, Approximation by superpositions of a sigmoidal
function, Mathematics of Control Signals and Systems \textbf{2}, pp.
303-314, 1989.

\bibitem{rbf2}H. Yu, T. Xie, S. Paszczynski, B. M. Wilamowski, Advantages
of Radial Basis Function Networks for Dynamic System Design, in IEEE
Transactions on Industrial Electronics \textbf{58}, pp. 5438-5450,
2011.

\bibitem{rbf3}D. Chen, Research on Traffic Flow Prediction in the
Big Data Environment Based on the Improved RBF Neural Network, IEEE
Transactions on Industrial Informatics \textbf{13}, pp. 2000-2008,
2017.

\bibitem{rbf4}Z. Yang, M. Mourshed, K. Liu, X. Xu, S. Feng, A novel
competitive swarm optimized RBF neural network model for short-term
solar power generation forecasting, Neurocomputing \textbf{397}, pp.
415-421, 2020.

\bibitem{svm}I. Steinwart, A. Christmann, Support Vector Machines,
Information Science and Statistics, Springer, 2008.

\bibitem{svm2}A. Iranmehr, H. Masnadi-Shirazi, N. Vasconcelos, Cost-sensitive
support vector machines, Neurocomputing \textbf{343}, pp. 50-64, 2019.

\bibitem{dt1}S.B. Kotsiantis, Decision trees: a recent overview,
Artif Intell Rev \textbf{39}, pp. 261--283, 2013.

\bibitem{dt2}D. Bertsimas, J. Dunn, Optimal classification trees,
Mach Learn \textbf{106}, pp. 1039--1082, 2017.

\bibitem{class_review}S.B. Kotsiantis, I.D. Zaharakis, P.E. Pintelas,
Machine learning: a review of classification and combining techniques,
Artif Intell Rev \textbf{26}, pp. 159--190, 2006.

\bibitem{bpnn}D.E. Rumelhart, G.E. Hinton and R.J. Williams, Learning
representations by back-propagating errors, Nature \textbf{323}, pp.
533 - 536 , 1986.

\bibitem{bpnn2}T. Chen and S. Zhong, Privacy-Preserving Backpropagation
Neural Network Learning, IEEE Transactions on Neural Networks \textbf{20},
, pp. 1554-1564, 2009.

\bibitem{geneticnn1}F. H. F. Leung, H. K. Lam, S. H. Ling and P.
K. S. Tam, Tuning of the structure and parameters of a neural network
using an improved genetic algorithm, IEEE Transactions on Neural Networks
\textbf{14}, pp. 79-88, 2003.

\bibitem{geneticnn2}A. Sedki, D. Ouazar, E. El Mazoudi, Evolving
neural network using real coded genetic algorithm for daily rainfall--runoff
forecasting, Expert Systems with Applications \textbf{36}, pp. 4523-4527,
2009.

\bibitem{geneticnn3}A. Majdi, M. Beiki, Evolving neural network using
a genetic algorithm for predicting the deformation modulus of rock
masses, International Journal of Rock Mechanics and Mining Sciences
\textbf{47}, pp. 246-253, 2010.

\bibitem{Powell}M.J.D Powell, A Tolerant Algorithm for Linearly Constrained
Optimization Calculations, Mathematical Programming \textbf{45}, pp.
547-566, 1989. 

\bibitem{nndimension}M. Verleysen, D. Francois, G. Simon, V. Wertz,
On the effects of dimensionality on data analysis with neural networks.
In: Mira J., Álvarez J.R. (eds) Artificial Neural Nets Problem Solving
Methods. IWANN 2003. Lecture Notes in Computer Science, vol 2687.
Springer, Berlin, Heidelberg. 2003.

\bibitem{nnpca1}Burcu Erkmen, Tülay Yıldırım, Improving classification
performance of sonar targets by applying general regression neural
network with PCA, Expert Systems with Applications \textbf{35}, pp.
472-475, 2008.

\bibitem{nnpca2}Jing Zhou, Aihuang Guo, Branko Celler, Steven Su,
Fault detection and identification spanning multiple processes by
integrating PCA with neural network, Applied Soft Computing \textbf{14},
pp. 4-11, 2014.

\bibitem{nnpca3}Ravi Kumar G., Nagamani K., Anjan Babu G., A Framework
of Dimensionality Reduction Utilizing PCA for Neural Network Prediction.
In: Borah S., Emilia Balas V., Polkowski Z. (eds) Advances in Data
Science and Management. Lecture Notes on Data Engineering and Communications
Technologies, vol 37. Springer, Singapore. 2020.

\bibitem{mrmr1}Hanchuan Peng, Fuhui Long, and Chris Ding, Feature
selection based on mutual information: criteria of max-dependency,
max-relevance, and min-redundancy, IEEE Transactions on Pattern Analysis
and Machine Intelligence \textbf{27}, pp.1226-1238, 2005.

\bibitem{mrmr2}Chris Ding, and Hanchuan Peng, Minimum redundancy
feature selection from microarray gene expression data, Journal of
Bioinformatics and Computational Biology \textbf{3}, pp.185-205, 2005.

\bibitem{nn_autoencoder}Y. Wang, H. Yao, S. Zhao, Auto-encoder based
dimensionality reduction, Neurocomputing \textbf{184}, pp. 232-242,
2016.

\bibitem{nngeman}S. Geman, E. Bienenstock and R. Doursat, Neural
networks and the bias/variance dilemma, Neural Computation 4 , pp.
1 - 58, 1992.

\bibitem{nnhawkins}Douglas M. Hawkins, The Problem of Overfitting,
J. Chem. Inf. Comput. Sci. \textbf{44}, pp. 1--12, 2004.

\bibitem{nnsharing1}S.J. Nowlan and G.E. Hinton, Simplifying neural
networks by soft weight sharing, Neural Computation 4, pp. 473-493,
1992.

\bibitem{nnsharing2}W. Roth, F. Pernkopf, Bayesian Neural Networks
with Weight Sharing Using Dirichlet Processes, IEEE Transactions on
Pattern Analysis and Machine Intelligence \textbf{42}, pp. 246-252,
2020.

\bibitem{nnprunning1}S.J. Hanson and L.Y. Pratt, Comparing biases
for minimal network construction with back propagation, In D.S. Touretzky
(Ed.), Advances in Neural Information Processing Systems, Volume 1,
pp. 177-185, San Mateo, CA: Morgan Kaufmann, 1989.

\bibitem{nnprunning2}M.C. Mozer and P. Smolensky, Skeletonization:
a technique for trimming the fat from a network via relevance assesment.
In D.S. Touretzky (Ed.), Advances in Neural Processing Systems, Volume
1, pp. 107-115, San Mateo CA: Morgan Kaufmann, 1989.

\bibitem{nnprunning3}M. Augasta and T. Kathirvalavakumar, Pruning
algorithms of neural networks --- a comparative study, Central European
Journal of Computer Science, 2003.

\bibitem{nnelim1}F. Hergert, W. Finnoff, H. G. Zimmermann, A comparison
of weight elimination methods for reducing complexity in neural networks,
In: Proceedings 1992{]} IJCNN International Joint Conference on Neural
Networks, Baltimore, MD, USA, pp. 980-987 vol.3, 1992.

\bibitem{nnelim2}M. Cottrell, B. Girard, Y. Girard, M. Mangeas, C.
Muller, Neural modeling for time series: A statistical stepwise method
for weight elimination, IEEE Transactions on Neural Networks \textbf{6},
pp. 1355-1364, 1995.

\bibitem{nnelim3}C. M. Ennett, M. Frize, Weight-elimination neural
networks applied to coronary surgery mortality prediction, IEEE Transactions
on Information Technology in Biomedicine \textbf{7}, pp. 86-92, 2003.

\bibitem{nndecay1}N. K. Treadgold and T. D. Gedeon, Simulated annealing
and weight decay in adaptive learning: the SARPROP algorithm,IEEE
Transactions on Neural Networks \textbf{9}, pp. 662-668, 1998.

\bibitem{nndecay2}M. Carvalho and T. B. Ludermir, Particle Swarm
Optimization of Feed-Forward Neural Networks with Weight Decay, 2006
Sixth International Conference on Hybrid Intelligent Systems (HIS'06),
Rio de Janeiro, Brazil, 2006, pp. 5-5.

\bibitem{ge1}M. O’Neill, C. Ryan, Grammatical evolution, IEEE Trans.
Evol. Comput. \textbf{5,}pp. 349--358, 2001.

\bibitem{geApp1}Alfonso Ortega, Rafael Sánchez, Manuel Alfonseca
Moreno, Automatic composition of music by means of grammatical evolution,
APL '02 Proceedings of the 2002 conference on APL: array processing
languages: lore, problems, and applications Pages 148 - 155.

\bibitem{geApp2}Michael O’Neill, Anthony Brabazon, Conor Ryan, J.
J. Collins, Evolving Market Index Trading Rules Using Grammatical
Evolution, Applications of Evolutionary Computing Volume 2037 of the
series Lecture Notes in Computer Science pp 343-352. 

\bibitem{geApp3}M. O’Neill, C. Ryan, Grammatical Evolution: Evolutionary
Automatic Programming in a Arbitary Language, Genetic Programming,
vol. 4, Kluwer Academic Publishers, Dordrecht, 2003

\bibitem{geApp4}J.J. Collins, C. Ryan, in: Proceedings of AROB 2000,
Fifth International Symposium on Artificial Life and Robotics, 2000.

\bibitem{geApp5}M. O’Neill, C. Ryan, in: K. Miettinen, M.M. Mkel,
P. Neittaanmki, J. Periaux (Eds.), Evolutionary Algorithms in Engineering
and Computer Science, Jyvskyl, Finland, 1999, pp. 127--134.

\bibitem{pso_major}J. Kennedy and R. Eberhart, \textquotedbl Particle
swarm optimization,\textquotedbl{} Proceedings of ICNN'95 - International
Conference on Neural Networks, 1995, pp. 1942-1948 vol.4, doi: 10.1109/ICNN.1995.488968.

\bibitem{pso1}Riccardo Poli, James Kennedy kennedy, Tim Blackwell,
Particle swarm optimization An Overview, Swarm Intelligence \textbf{1},
pp 33-57, 2007. 

\bibitem{pso2}Ioan Cristian Trelea, The particle swarm optimization
algorithm: convergence analysis and parameter selection, Information
Processing Letters \textbf{85}, pp. 317-325, 2003.

\bibitem{psophysics1}Anderson Alvarenga de Moura Meneses, Marcelo
Dornellas, Machado Roberto Schirru, Particle Swarm Optimization applied
to the nuclear reload problem of a Pressurized Water Reactor, Progress
in Nuclear Energy \textbf{51}, pp. 319-326, 2009.

\bibitem{psophysics2}Ranjit Shaw, Shalivahan Srivastava, Particle
swarm optimization: A new tool to invert geophysical data, Geophysics
\textbf{72}, 2007.

\bibitem{psochem1}C. O. Ourique, E.C. Biscaia, J.C. Pinto, The use
of particle swarm optimization for dynamical analysis in chemical
processes, Computers \& Chemical Engineering \textbf{26}, pp. 1783-1793,
2002.

\bibitem{psochem2}H. Fang, J. Zhou, Z. Wang et al, Hybrid method
integrating machine learning and particle swarm optimization for smart
chemical process operations, Front. Chem. Sci. Eng. \textbf{16}, pp.
274--287, 2022.

\bibitem{psomed1}M.P. Wachowiak, R. Smolikova, Yufeng Zheng, J.M.
Zurada, A.S. Elmaghraby, An approach to multimodal biomedical image
registration utilizing particle swarm optimization, IEEE Transactions
on Evolutionary Computation \textbf{8}, pp. 289-301, 2004.

\bibitem{psomed2}Yannis Marinakis. Magdalene Marinaki, Georgios Dounias,
Particle swarm optimization for pap-smear diagnosis, Expert Systems
with Applications \textbf{35}, pp. 1645-1656, 2008. 

\bibitem{psoecon}Jong-Bae Park, Yun-Won Jeong, Joong-Rin Shin, Kwang
Y. Lee, An Improved Particle Swarm Optimization for Nonconvex Economic
Dispatch Problems, IEEE Transactions on Power Systems \textbf{25},
pp. 156-16\textbf{216}6, 2010.

\bibitem{psoApp1}B. Liu, L. Wang, Y.H. Jin, An Effective PSO-Based
Memetic Algorithm for Flow Shop Scheduling, IEEE Transactions on Systems,
Man, and Cybernetics, Part B (Cybernetics) \textbf{37}, pp. 18-27,
2007.

\bibitem{psoApp2}J. Yang, L. He, S. Fu, An improved PSO-based charging
strategy of electric vehicles in electrical distribution grid, Applied
Energy \textbf{128}, pp. 82-92, 2014.

\bibitem{psoApp3}K. Mistry, L. Zhang, S. C. Neoh, C. P. Lim, B. Fielding,
A Micro-GA Embedded PSO Feature Selection Approach to Intelligent
Facial Emotion Recognition, IEEE Transactions on Cybernetics. \textbf{47},
pp. 1496-1509, 2017.

\bibitem{psoApp4}S. Han, X. Shan, J. Fu, W. Xu, H. Mi, Industrial
robot trajectory planning based on improved pso algorithm, J. Phys.:
Conf. Ser. \textbf{1820}, 012185, 2021.

\bibitem{fc1}Dimitris Gavrilis, Ioannis G. Tsoulos, Evangelos Dermatas,
Selecting and constructing features using grammatical evolution, Pattern
Recognition Letters \textbf{29},pp. 1358-1365, 2008. 

\bibitem{fc2}Dimitris Gavrilis, Ioannis G. Tsoulos, Evangelos Dermatas,
Neural Recognition and Genetic Features Selection for Robust Detection
of E-Mail Spam, Advances in Artificial Intelligence Volume 3955 of
the series Lecture Notes in Computer Science pp 498-501, 2006.

\bibitem{fc3}George Georgoulas, Dimitris Gavrilis, Ioannis G. Tsoulos,
Chrysostomos Stylios, João Bernardes, Peter P. Groumpos, Novel approach
for fetal heart rate classification introducing grammatical evolution,
Biomedical Signal Processing and Control \textbf{2},pp. 69-79, 2007 

\bibitem{fc4}Otis Smart, Ioannis G. Tsoulos, Dimitris Gavrilis, George
Georgoulas, Grammatical evolution for features of epileptic oscillations
in clinical intracranial electroencephalograms, Expert Systems with
Applications \textbf{38}, pp. 9991-9999, 2011.

\bibitem{fc5}I.G. Tsoulos, C. Stylios, V. Charalampous, COVID-19
Predictive Models Based on Grammatical Evolution, SN COMPUT. SCI.
\textbf{4}, 191, 2023.

\bibitem{fc6}V. Christou, I.G. Tsoulos, V. Loupas, A.T. Tzallas,
C. Gogos, P.S. Karvelis, N. Antoniadis, E. Glavas, N. Giannakeas,
Performance and early drop prediction for higher education students
using machine learning, Expert Systems with Applications \textbf{225},
120079, 2023.

\bibitem{fsel1}A. Verikas, M. Bacauskiene, Feature selection with
neural networks, Pattern Recognition Letters \textbf{23}, pp. 1323-1335,
2002.

\bibitem{fsel2}Md. Monirul Kabir, Md. Monirul Islam, K. Murase, A
new wrapper feature selection approach using neural network, Neurocomputing
\textbf{73}, pp. 3273-3283, 2010.

\bibitem{genfc_anneal}V.S. Devi, Class Specific Feature Selection
Using Simulated Annealing. In: Prasath, R., Vuppala, A., Kathirvalavakumar,
T. (eds) Mining Intelligence and Knowledge Exploration. MIKE 2015.
Lecture Notes in Computer Science(), vol 9468. Springer, Cham. 2015.

\bibitem{genfc2}K. Neshatian, M. Zhang, P. Andreae, A Filter Approach
to Multiple Feature Construction for Symbolic Learning Classifiers
Using Genetic Programming, IEEE Transactions on Evolutionary Computation
\textbf{16}, pp. 645-661, 2012.

\bibitem{bnf1}J. W. Backus. The Syntax and Semantics of the Proposed
International Algebraic Language of the Zurich ACM-GAMM Conference.
Proceedings of the International Conference on Information Processing,
UNESCO, 1959, pp.125-132.

\bibitem{Keel}J. Alcalá-Fdez, A. Fernandez, J. Luengo, J. Derrac,
S. García, L. Sánchez, F. Herrera. KEEL Data-Mining Software Tool:
Data Set Repository, Integration of Algorithms and Experimental Analysis
Framework. Journal of Multiple-Valued Logic and Soft Computing 17,
pp. 255-287, 2011.

\bibitem{appendicitis}Weiss, Sholom M. and Kulikowski, Casimir A.,
Computer Systems That Learn: Classification and Prediction Methods
from Statistics, Neural Nets, Machine Learning, and Expert Systems,
Morgan Kaufmann Publishers Inc, 1991.

\bibitem{appendicitis2}M. Wang, Y.Y. Zhang, F. Min, Active learning
through multi-standard optimization, IEEE Access \textbf{7}, pp. 56772--56784,
2019.

\bibitem{australian}J.R. Quinlan, Simplifying Decision Trees. International
Journal of Man-Machine Studies \textbf{27}, pp. 221-234, 1987. 

\bibitem{balance}T. Shultz, D. Mareschal, W. Schmidt, Modeling Cognitive
Development on Balance Scale Phenomena, Machine Learning \textbf{16},
pp. 59-88, 1994.

\bibitem{bands}B. Evans, D. Fisher, Overcoming process delays with
decision tree induction. IEEE Expert \textbf{9}, pp. 60-66, 1994.

\bibitem{dermatology}G. Demiroz, H.A. Govenir, N. Ilter, Learning
Differential Diagnosis of Eryhemato-Squamous Diseases using Voting
Feature Intervals, Artificial Intelligence in Medicine. \textbf{13},
pp. 147--165, 1998.

\bibitem{heart}I. Kononenko, E. Šimec, M. Robnik-Šikonja, Overcoming
the Myopia of Inductive Learning Algorithms with RELIEFF, Applied
Intelligence \textbf{7}, pp. 39--55, 1997

\bibitem{hayesroth}B. Hayes-Roth, B., F. Hayes-Roth. Concept learning
and the recognition and classification of exemplars. Journal of Verbal
Learning and Verbal Behavior \textbf{16}, pp. 321-338, 1977.

\bibitem{housevotes}R.M. French, N. Chater, Using noise to compute
error surfaces in connectionist networks: a novel means of reducing
catastrophic forgetting, Neural Comput. \textbf{14}, pp. 1755-1769,
2002.

\bibitem{ion1}J.G. Dy , C.E. Brodley, Feature Selection for Unsupervised
Learning, The Journal of Machine Learning Research \textbf{5}, pp
845--889, 2004.

\bibitem{ion2}S. J. Perantonis, V. Virvilis, Input Feature Extraction
for Multilayered Perceptrons Using Supervised Principal Component
Analysis, Neural Processing Letters \textbf{10}, pp 243--252, 1999.

\bibitem{liver1}J. Mcdermott, R.S. Forsyth, Diagnosing a disorder
in a classification benchmark, Pattern Recognition Letters \textbf{73},
pp. 41-43, 2016.

\bibitem{liver2} J. Garcke, M. Griebel, Classification with sparse
grids using simplicial basis functions, Intell. Data Anal. \textbf{6},
pp. 483-502, 2002.

\bibitem{mammographic}M. Elter, R. Schulz-Wendtland, T. Wittenberg,
The prediction of breast cancer biopsy outcomes using two CAD approaches
that both emphasize an intelligible decision process, Med Phys. \textbf{34},
pp. 4164-72, 2007.

\bibitem{parkinsons1}M.A. Little, P.E. McSharry, S.J Roberts et al,
Exploiting Nonlinear Recurrence and Fractal Scaling Properties for
Voice Disorder Detection. BioMed Eng OnLine \textbf{6}, 23, 2007.

\bibitem{parkinsons2}M.A. Little, P.E. McSharry, E.J. Hunter, J.
Spielman, L.O. Ramig, Suitability of dysphonia measurements for telemonitoring
of Parkinson's disease. IEEE Trans Biomed Eng. \textbf{56}, pp. 1015-1022,
2009.

\bibitem{pima}J.W. Smith, J.E. Everhart, W.C. Dickson, W.C. Knowler,
R.S. Johannes, Using the ADAP learning algorithm to forecast the onset
of diabetes mellitus, In: Proceedings of the Symposium on Computer
Applications and Medical Care IEEE Computer Society Press, pp.261-265,
1988.

\bibitem{popfailures}D.D. Lucas, R. Klein, J. Tannahill, D. Ivanova,
S. Brandon, D. Domyancic, Y. Zhang, Failure analysis of parameter-induced
simulation crashes in climate models, Geoscientific Model Development
\textbf{6}, pp. 1157-1171, 2013.

\bibitem{regions}N. Giannakeas, M.G. Tsipouras, A.T. Tzallas, K.
Kyriakidi, Z.E. Tsianou, P. Manousou, A. Hall, E.C. Karvounis, V.
Tsianos, E. Tsianos, A clustering based method for collagen proportional
area extraction in liver biopsy images (2015) Proceedings of the Annual
International Conference of the IEEE Engineering in Medicine and Biology
Society, EMBS, 2015-November, art. no. 7319047, pp. 3097-3100. 

\bibitem{saheart}T. Hastie, R. Tibshirani, Non-parametric logistic
and proportional odds regression, JRSS-C (Applied Statistics) \textbf{36},
pp. 260--276, 1987.

\bibitem{segment}M. Dash, H. Liu, P. Scheuermann, K. L. Tan, Fast
hierarchical clustering and its validation, Data \& Knowledge Engineering
\textbf{44}, pp 109--138, 2003.

\bibitem{wdbc}W.H. Wolberg, O.L. Mangasarian, Multisurface method
of pattern separation for medical diagnosis applied to breast cytology,
Proc Natl Acad Sci U S A. \textbf{87}, pp. 9193--9196, 1990.

\bibitem{wine1}M. Raymer, T.E. Doom, L.A. Kuhn, W.F. Punch, Knowledge
discovery in medical and biological datasets using a hybrid Bayes
classifier/evolutionary algorithm. IEEE transactions on systems, man,
and cybernetics. Part B, Cybernetics : a publication of the IEEE Systems,
Man, and Cybernetics Society, \textbf{33} , pp. 802-813, 2003.

\bibitem{wine2}P. Zhong, M. Fukushima, Regularized nonsmooth Newton
method for multi-class support vector machines, Optimization Methods
and Software \textbf{22}, pp. 225-236, 2007.

\bibitem{eeg1}R. G. Andrzejak, K. Lehnertz, F.Mormann, C. Rieke,
P. David, and C. E. Elger, “Indications of nonlinear deterministic
and finite-dimensional structures in time series of brain electrical
activity: dependence on recording region and brain state,” Physical
Review E, vol. 64, no. 6, Article ID 061907, 8 pages, 2001. 

\bibitem{eeg2}A. T. Tzallas, M. G. Tsipouras, and D. I. Fotiadis,
“Automatic Seizure Detection Based on Time-Frequency Analysis and
Artificial Neural Networks,” Computational Intelligence and Neuroscience,
vol. 2007, Article ID 80510, 13 pages, 2007. doi:10.1155/2007/80510

\bibitem{zoo}M. Koivisto, K. Sood, Exact Bayesian Structure Discovery
in Bayesian Networks, The Journal of Machine Learning Research\textbf{
5}, pp. 549--573, 2004.

\bibitem{key-20}R.G. Andrzejak, K. Lehnertz, F. Mormann, C. Rieke,
P. David, and C. E. Elger, Indications of nonlinear deterministic
and finite-dimensional structures in time series of brain electrical
activity: Dependence on recording region and brain state, Phys. Rev.
E \textbf{64}, pp. 1-8, 2001.

\bibitem{abalone}W. J Nash, T.L. Sellers, S.R. Talbot, A.J. Cawthor,
W.B. Ford, The Population Biology of Abalone (\_Haliotis\_ species)
in Tasmania. I. Blacklip Abalone (\_H. rubra\_) from the North Coast
and Islands of Bass Strait, Sea Fisheries Division, Technical Report
No. 48 (ISSN 1034-3288), 1994.

\bibitem{airfoil}T.F. Brooks, D.S. Pope, A.M. Marcolini, Airfoil
self-noise and prediction. Technical report, NASA RP-1218, July 1989. 

\bibitem{Stat}J.S. Simonoff, Smooting Methods in Statistics, Springer
- Verlag, 1996.

\bibitem{concrete}I.Cheng Yeh, Modeling of strength of high performance
concrete using artificial neural networks, Cement and Concrete Research.
\textbf{28}, pp. 1797-1808, 1998. 

\bibitem{key23}D. Harrison and D.L. Rubinfeld, Hedonic prices and
the demand for clean ai, J. Environ. Economics \& Management \textbf{5},
pp. 81-102, 1978.

\bibitem{key21}J.S. Simonoff, Smooting Methods in Statistics, Springer
- Verlag, 1996.

\bibitem{pydataset}R.D. King, S. Muggleton, R. Lewis, M.J.E. Sternberg,
Proc. Nat. Acad. Sci. USA \textbf{89}, pp. 11322--11326, 1992. 

\bibitem{bfgs2}R. Fletcher, A new approach to variable metric algorithms,
Computer Journal \textbf{13}, pp. 317-322, 1970. 

\bibitem{rbf1}J. Park and I. W. Sandberg, Universal Approximation
Using Radial-Basis-Function Networks, Neural Computation \textbf{3},
pp. 246-257, 1991.

\bibitem{Adam}D. P. Kingma, J. L. Ba, ADAM: a method for stochastic
optimization, in: Proceedings of the 3rd International Conference
on Learning Representations (ICLR 2015), pp. 1--15, 2015.

\bibitem{rpropnn}M. Riedmiller and H. Braun, A Direct Adaptive Method
for Faster Backpropagation Learning: The RPROP algorithm, Proc. of
the IEEE Intl. Conf. on Neural Networks, San Francisco, CA, pp. 586--591,
1993.

\bibitem{rpropnn3}T. Pajchrowski, K. Zawirski and K. Nowopolski,
Neural Speed Controller Trained Online by Means of Modified RPROP
Algorithm, IEEE Transactions on Industrial Informatics \textbf{11},
pp. 560-568, 2015.

\bibitem{rpropnn2}Rinda Parama Satya Hermanto, Suharjito, Diana,
Ariadi Nugroho, Waiting-Time Estimation in Bank Customer Queues using
RPROP Neural Networks, Procedia Computer Science \textbf{ 135}, pp.
35-42, 2018.

\bibitem{neat}K. O. Stanley, R. Miikkulainen, Evolving Neural Networks
through Augmenting Topologies, Evolutionary Computation \textbf{10},
pp. 99-127, 2002.

\end{thebibliography}

\end{adjustwidth}{}
\end{document}
